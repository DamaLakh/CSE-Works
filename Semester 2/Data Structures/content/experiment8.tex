\section{Queue}
\label{sec:Queue}

\subsection{Objective}
Write a program to implement a queue using an array

\subsection{Algorithm}
\begin{lstlisting}[style=mystyle]
Queue operations work as follows:

    two pointers FRONT and REAR
    FRONT track the first element of the queue
    REAR track the last element of the queue
    initially, set value of FRONT and REAR to -1

Enqueue Operation

    check if the queue is full
    for the first element, set the value of FRONT to 0
    increase the REAR index by 1
    add the new element in the position pointed to by REAR

Dequeue Operation

    check if the queue is empty
    return the value pointed by FRONT
    increase the FRONT index by 1
    for the last element, reset the values of FRONT and REAR to -1
\end{lstlisting}

\subsection{Code}
\inputminted[]{c}{../../Code/queue.c}

\subsection{Output}
\begin{lstlisting}[style=output]
Queue is Empty!!
Inserted -> 1
Inserted -> 2
Inserted -> 3
Inserted -> 4
Inserted -> 5
Queue is Full!!
Queue elements are:
1  2  3  4  5  

Deleted : 1
Queue elements are:
2  3  4  5 
\end{lstlisting}

\pagebreak
