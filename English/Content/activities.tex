\section{Self Introduction}

\subsection{Objective}

\begin{enumerate}
    \item To enable oneself to begin a conversation through the usage of an introduction.
    \item To encourage oneself to explore aspects of their own life that contribute to their identity.
    \item Seek information about others by using correct sentences.

\end{enumerate}

\subsection{Methodology}

\begin{enumerate}
    \item Listen to the shared audio or video related to self-introduction.
    \item Prepare the list of points that we have to include in the self-introduction.
    \item Prepare the content of the self-introduction concisely.
    \item Then introduce yourself in front of a group of people.
\end{enumerate}

\subsection{Activity}

\subsection{Learning Outcomes}

\begin{enumerate}
    \item Confidence building was encouraged.
    \item Usage of appropriate tone and phraseology was learned.
    \item Communication skills were enhanced.
    \item We were able to learn how to start a conversation with anyone through self-introduction.
\end{enumerate}

\pagebreak

\section{Phonetics}

\subsection{Objective}
\begin{enumerate}
    \item Understand the system of sound and sound combinations in English (Phonology).
    \item Understand how sounds are produced, how they are transmitted, and how they are
          perceived (Phonetics).
    \item Differentiate between consonants and vowel sounds in all word positions.
    \item Pronounce English sounds in isolation and in connected speech.
\end{enumerate}

\subsection{Methodology}
\begin{enumerate}
    \item First, we learned about the IPA symbols of 20 vowel sounds.
    \item We learned about the two basic categories of vowel sounds, namely, monophthongs
          and diphthongs.
    \item Then, we learned about the IPA symbols of 22 sounds available in the consonant
          category.
    \item We understood these vowel and consonant sounds better with the help of some
          examples.
    \item After learning the sounds, we moved on to transcribe different words for each sound.
\end{enumerate}

\subsection{Activity}

\subsection{Learning Outcomes}
\begin{enumerate}
    \item We became familiar with the basic category of vowels and constants on the basis of
          sounds.
    \item We were able to recognize many of the sounds of the IPA chart and the parameters
          according to which sounds can vary, and describe them using appropriate terminology
          and symbolization.
    \item Furthermore, we were able to produce simple phonetic descriptions and broad phonetic
          transcriptions of short stretches of speech.
\end{enumerate}
\pagebreak

\section{Reading From Newspapers/ Magazines To Build Up A Repertoire Of Words}

\subsection{Objective}
\begin{enumerate}
    \item To undertake extensive and independent reading of newspapers and magazines to
          expand word knowledge.
    \item To get knowledge about the meaning and usage of that particular word in a sentence.
    \item Learn about the antonyms and synonyms of that particular word.
    \item Gain knowledge about the proper pronunciation of that particular word.
\end{enumerate}

\subsection{Methodology}
\begin{enumerate}
    \item Reading any newspaper or magazine on a daily basis for one month.
    \item Underlining new words present in the text for which we had no prior knowledge.
    \item Search for the meaning, antonym, synonyms of that particular word.
    \item See the usage of that word in some example sentences.
\end{enumerate}

\subsection{Activity}

\subsection{Learning Outcomes}
\begin{enumerate}
    \item It helps improve our understanding of novels and textbooks.
    \item It results in better communication skills.
    \item Helped us in expressing ourselves better in our writings.
\end{enumerate}

\pagebreak

\section{Roleplay}

\subsection{Objective}
\begin{enumerate}
    \item To learn, improve and develop the skills or competencies necessary for a specific
          position.
    \item To learn and understand the roles of stakeholders in particular situations.
    \item Learn proper characterization of ideas spontaneously.
    \item Utilize the English language as the only means of communication throughout the
          narrative.
    \item Develops social skills by learning to collaborate with others and work as a team.
\end{enumerate}

\subsection{Methodology}
\begin{enumerate}
    \item Groups of students were formed for the roleplay.
    \item Proper scene or act for the play was decided.
    \item Script for the roleplay was written.
    \item Each member of the group was assigned their role.
    \item Play was enacted by all the participants of the group.
\end{enumerate}

\subsection{Activity}

\subsection{Learning Outcomes}
\begin{enumerate}
    \item It helped each participant to learn the role or tasks of a job by practicing or simulating
          real working conditions.
    \item It helped in building confidence for speaking anytime and anywhere with ease and
          clarity.
    \item Helped develop listening skills, since, for a good role play, members should be able
          to comprehend each other's thoughts.
    \item Creative problem-solving is learned since the members learned to discuss and solve
          problems regarding a situation on the spot.
\end{enumerate}

\pagebreak

\section{Question Formation \& Mock Press Conference}

\subsection{Objective}
\begin{enumerate}
    \item To explore and understand the role of a press conference in gathering news and in
          disseminating news.
    \item To brainstorm about current issues within a group.
    \item Write and edit a feature news story based on information revealed at the press
          conference.
    \item To evaluate the thinking abilities of a person.
\end{enumerate}

\subsection{Methodology}
\begin{enumerate}
    \item We formed a group of some students for a press conference.
    \item Then, we decided on the purpose of the conference.
    \item Then, we divided the roles between the members.
    \item After that, we gathered information about the purpose of the conference such as the
          kind of questions that could be asked and the possible answers to those questions.
    \item Finally, we started writing questions and their answers as a team.
\end{enumerate}

\subsection{Activity}

\subsection{Learning Outcomes}
\begin{enumerate}
    \item It helped gather a lot of information about something by writing questions and then
          answering them.
    \item It helped build thinking and speaking skills.
    \item Helped us in understanding how to work as a team.
    \item It helped us in understanding how to exchange ideas among group members in a
          structured and organized manner.
\end{enumerate}

\pagebreak
